\documentclass[12pt, a4paper]{article}
\usepackage{scrextend}
\usepackage{titlesec}
\usepackage{graphicx}
\usepackage{amsmath}
\usepackage{amsfonts} % for the real number symbol
\usepackage{geometry}
\usepackage[unicode]{hyperref}
\usepackage{titlesec}
\usepackage{titletoc}
\usepackage[sorting=none]{biblatex}
\usepackage{xurl}
\usepackage{enumitem}
\usepackage{indentfirst}
\numberwithin{equation}{section} % number equations by section
\renewcommand{\figurename}{Att.}
\renewcommand{\contentsname}{Saturs}
\renewcommand{\labelenumi}{\arabic{enumi})} % lists with 1)
\setlist{nosep}
\parindent=1cm
\linespread{1.213} % equivalent to 1.5 in word, experimentally.
\addbibresource{refs.bib}

\geometry{
    a4paper,
    lmargin=30mm,
    rmargin=20mm,
    tmargin=20mm,
    bmargin=20mm
}


\titleformat{\section}
    {\normalfont\large\bfseries}{\thesection . }{0.2em}{\MakeUppercase}
\titleformat{\subsection}
    {\normalfont\large\bfseries}{\thesubsection . }{0.2em}{}
\titleformat{\subsubsection}
    {\normalfont\normalsize\bfseries\itshape}{\thesubsubsection . }{0.2em}{}
\titlespacing*{\subsubsection}{0pt}{6pt}{0pt}

\begin{document}
\begin{titlepage}
    \begin{center}
        \vspace*{3cm}
        
        LATVIJAS UNIVERSITĀTE

        DATORIKAS FAKULTĀTE

        \vspace*{4cm}

        \large\textbf{COVID-19 IEROBEŽOŠANAS PROJEKTU PĀRVALDĪBAS ASPEKTI}
        
        \vspace{2cm}
        \normalsize{REFERĀTS IT PROJEKTU PĀRVALDĪBĀ}
         
             
    \end{center}
    \vspace{3cm}
    \begin{addmargin}[18em]{0em}
    Autors: \textbf{Pēteris Račinskis}
    \end{addmargin}

    \begin{addmargin}[18em]{0em}
    \hspace{1cm} Stud. apl. Nr. pr20015
    \end{addmargin}

         
    \vfill
    \begin{center}
    RĪGA 2022
    \end{center}
 \end{titlepage}
\newpage
\tableofcontents
\thispagestyle{empty}
\newpage
\setcounter{page}{3}


\section{Ievads}

Pagājuši nedaudz vairāk kā trīs gadi, kopš parādījās pirmās ziņas, ka Uhaņā --- vienām no Ķīnas lielākajām pilsētām --- dažiem pacientiem konstatēta saslimšana ar iepriekš neredzētu infekcijas slimību. Tās simptomi atgādina gripu, taču vīruss nepieder pie gripas vīrusu dzimtes --- tas ir jauns, potenciāli ārkārtīgi lipīgs koronavīruss. Iepriekšējā reize, kad Ķīnā atklāts cilvēkam bīstams koronavīruss, bijusi 2002. gada rudenī, kad sākusies nāvējošā, taču mērogā samērā ierobežotā SARS epidēmija. Sociālajos tīklos šīs ziņas strauji sasniedz visu pasauli, medicīnas aprindās virmo satraukums, taču rietumos plašsaziņas līdzekļi un sabiedrība kopumā lielu uzmanību tām nepievērš --- jaunu, bīstamu vīrusu parādīšanās kaut kur tālu prom ir regulāra parādība, kas, par spīti dažādu pasaules gala solītāju vaimanāšanai, nekad taču pie nopietnām sekām šeit pie mums nenovedīs. 

Reti kurš toreiz, ap 2019/2020. gadu miju spēja iztēloties to, ka tikai dažus mēnešus vēlāk lielveikalu plauktus panikā tukšos ar tualetes papīru apkrāvušos cilvēku drūzmas un ziņas par miljonu nāvi būs vien fona troksnis, bet šobrīd --- 3 gadus vēlāk --- sadzīvošana ar epidemioloģiskajiem ierobežojumiem ir kļuvusi par ikdienu, vakcinācija --- par karstāko tematu politikā --- un grūti vairs atcerēties, kā dzīve norita pirms tam. Teikt, ka pēdējie daži gadi ir bijuši visnotaļ interesanti valdībām un nevalstiskajām institūcijām, kuru uzdevumis bijis novērst, ierobežot un pārvarēt pandēmijas sekas, būtu maigi. Krīzes vadības stratēģijas un plāni, kas desmitgadēm ilgi bijušas vien teorētiski domas lidojumi, izvilktas no atvilknēm un liktas lietā. Iepriekš neredzēti investīciju apjomi ieguldīti tradicionāli ļoti lēno medicīnas izstrādes un sertificēšanas procesu paātrināšanā. Nācies praktiski saskarties un dārgi maksāt par nevīžīgi veidotiem sabiedriskās domas procesiem. 

Kaut gan pandēmija vēl nebūt nav galā, pagājis pietiekami ilgs laiks, lai varētu pakāpties soli atpakaļ, atskatīties uz šo īpatnējo vēstures periodu un mēģināt izdarīt kādus spriedumus. Kas īsti tika darīts? Kāpēc? Vai izdevās?

\subsection{COVID-19 pandēmija --- īss pārskats}


\subsection{Referāta mērķis un struktūra}

\newpage
\section{COVID-19 ierobežošanas projekti}

\subsection{Projektu dalījums}

\subsection{Vīrusa izplatības tūlītēja samazināšana}

\subsection{Vakcīnu izstrāde, ieviešana}

\subsection{Vakcinācija}


\newpage
\section{Secinājumi}

Literatūras analīzē sniegts īss un nebūt ne pilnīgs --- vai vienlīdzīgi sadalīts --- līdz šim par atdarinošo mašīnmācīšanos veiktās pētnieciskās darbības pārskats. Jau izvēloties, par kurām tēmām rakstīts plašāk, par kurām --- mazāk detalizēti --- iespaidu uz darba saturu ir atstājusi motivējošās problēmas specifika. Tagad nepieciešams pie tās atgriezties un novērtēt, kas no visa nozarē pētītā un izgudrotā attiecas uz darba ievadā aprakstīto uzdevumu --- mešanas kustību iestrādāšanu atkritumu vai citu objektu pārvietošanā.

\subsection{Svarīgākās atziņas}

\begin{figure}[t!]
    \centering
    \includegraphics[height=6.8cm,page=1]{../img/alvinn_architecture.png}
    \caption{ALVINN modeļa uzbūve \cite{enc_stim}}
\end{figure}

\newpage
\addcontentsline{toc}{section}{Atsauces}
\printbibliography[title=Atsauces]

\end{document}